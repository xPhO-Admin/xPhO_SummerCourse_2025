\begin{frame}{2 khối lượng - 3 lò xo}
    Hệ dao động liên kết cơ bản đầu tiên chúng ta tìm hiểu là hệ 2 khối lượng 3 lò xo. Bỏ qua độ dài tự nhiên của lò xo.
    \vspace{2mm}
    \begin{center}
        \scalebox{1}{


% Pattern Info
 
\tikzset{
pattern size/.store in=\mcSize, 
pattern size = 5pt,
pattern thickness/.store in=\mcThickness, 
pattern thickness = 0.3pt,
pattern radius/.store in=\mcRadius, 
pattern radius = 1pt}
\makeatletter
\pgfutil@ifundefined{pgf@pattern@name@_mysjemblw}{
\pgfdeclarepatternformonly[\mcThickness,\mcSize]{_mysjemblw}
{\pgfqpoint{0pt}{0pt}}
{\pgfpoint{\mcSize+\mcThickness}{\mcSize+\mcThickness}}
{\pgfpoint{\mcSize}{\mcSize}}
{
\pgfsetcolor{\tikz@pattern@color}
\pgfsetlinewidth{\mcThickness}
\pgfpathmoveto{\pgfqpoint{0pt}{0pt}}
\pgfpathlineto{\pgfpoint{\mcSize+\mcThickness}{\mcSize+\mcThickness}}
\pgfusepath{stroke}
}}
\makeatother

% Pattern Info
 
\tikzset{
pattern size/.store in=\mcSize, 
pattern size = 5pt,
pattern thickness/.store in=\mcThickness, 
pattern thickness = 0.3pt,
pattern radius/.store in=\mcRadius, 
pattern radius = 1pt}
\makeatletter
\pgfutil@ifundefined{pgf@pattern@name@_mggpvl8gd}{
\pgfdeclarepatternformonly[\mcThickness,\mcSize]{_mggpvl8gd}
{\pgfqpoint{0pt}{0pt}}
{\pgfpoint{\mcSize+\mcThickness}{\mcSize+\mcThickness}}
{\pgfpoint{\mcSize}{\mcSize}}
{
\pgfsetcolor{\tikz@pattern@color}
\pgfsetlinewidth{\mcThickness}
\pgfpathmoveto{\pgfqpoint{0pt}{0pt}}
\pgfpathlineto{\pgfpoint{\mcSize+\mcThickness}{\mcSize+\mcThickness}}
\pgfusepath{stroke}
}}
\makeatother
\tikzset{every picture/.style={line width=0.75pt}} %set default line width to 0.75pt        

\begin{tikzpicture}[x=0.75pt,y=0.75pt,yscale=-1,xscale=1]
%uncomment if require: \path (0,15225); %set diagram left start at 0, and has height of 15225

%Straight Lines [id:da3223058949903339] 
\draw    (446.5,1088.22) -- (86,1088.22) ;
%Shape: Rectangle [id:dp7648671038082127] 
\draw  [draw opacity=0][pattern=_mysjemblw,pattern size=6pt,pattern thickness=0.75pt,pattern radius=0pt, pattern color={rgb, 255:red, 0; green, 0; blue, 0}] (86,1047) -- (60.5,1047) -- (60.5,1088.22) -- (86,1088.22) -- cycle ;
%Shape: Resistor [id:dp6479321963022213] 
\draw   (86,1067.61) -- (102.65,1067.61) -- (106.35,1058.73) -- (113.75,1076.48) -- (121.15,1058.73) -- (128.55,1076.48) -- (135.95,1058.73) -- (143.35,1076.48) -- (150.75,1058.73) -- (158.15,1076.48) -- (161.85,1067.61) -- (178.5,1067.61) ;
%Shape: Square [id:dp15263822722006148] 
\draw   (178.5,1047) -- (219.72,1047) -- (219.72,1088.22) -- (178.5,1088.22) -- cycle ;
%Shape: Resistor [id:dp3406372469944945] 
\draw   (220,1067.61) -- (236.65,1067.61) -- (240.35,1058.73) -- (247.75,1076.48) -- (255.15,1058.73) -- (262.55,1076.48) -- (269.95,1058.73) -- (277.35,1076.48) -- (284.75,1058.73) -- (292.15,1076.48) -- (295.85,1067.61) -- (312.5,1067.61) ;
%Shape: Square [id:dp8671738733511334] 
\draw   (312.5,1047) -- (353.72,1047) -- (353.72,1088.22) -- (312.5,1088.22) -- cycle ;
%Shape: Resistor [id:dp4210341004898319] 
\draw   (354,1067.61) -- (370.65,1067.61) -- (374.35,1058.73) -- (381.75,1076.48) -- (389.15,1058.73) -- (396.55,1076.48) -- (403.95,1058.73) -- (411.35,1076.48) -- (418.75,1058.73) -- (426.15,1076.48) -- (429.85,1067.61) -- (446.5,1067.61) ;
%Straight Lines [id:da6091760378667466] 
\draw    (86,1047) -- (86,1088.22) ;
%Straight Lines [id:da1853906688587692] 
\draw    (446.5,1047) -- (446.5,1088.22) ;
%Shape: Rectangle [id:dp9695945528994532] 
\draw  [draw opacity=0][pattern=_mggpvl8gd,pattern size=6pt,pattern thickness=0.75pt,pattern radius=0pt, pattern color={rgb, 255:red, 0; green, 0; blue, 0}] (472,1047) -- (446.5,1047) -- (446.5,1088.22) -- (472,1088.22) -- cycle ;
%Straight Lines [id:da6417307585486507] 
\draw    (201,1118) -- (238.5,1118) ;
\draw [shift={(240.5,1118)}, rotate = 180] [color={rgb, 255:red, 0; green, 0; blue, 0 }  ][line width=0.75]    (10.93,-3.29) .. controls (6.95,-1.4) and (3.31,-0.3) .. (0,0) .. controls (3.31,0.3) and (6.95,1.4) .. (10.93,3.29)   ;
\draw [shift={(201,1118)}, rotate = 180] [color={rgb, 255:red, 0; green, 0; blue, 0 }  ][line width=0.75]    (0,5.59) -- (0,-5.59)   ;
%Straight Lines [id:da15068921372551158] 
\draw    (332,1118) -- (369.5,1118) ;
\draw [shift={(371.5,1118)}, rotate = 180] [color={rgb, 255:red, 0; green, 0; blue, 0 }  ][line width=0.75]    (10.93,-3.29) .. controls (6.95,-1.4) and (3.31,-0.3) .. (0,0) .. controls (3.31,0.3) and (6.95,1.4) .. (10.93,3.29)   ;
\draw [shift={(332,1118)}, rotate = 180] [color={rgb, 255:red, 0; green, 0; blue, 0 }  ][line width=0.75]    (0,5.59) -- (0,-5.59)   ;

% Text Node
\draw (127,1036.22) node [anchor=north west][inner sep=0.75pt]   [align=left] {$\displaystyle k_{1}$};
% Text Node
\draw (199.11,1067.61) node   [align=left] {$\displaystyle m_{1}$};
% Text Node
\draw (333.11,1067.61) node   [align=left] {$\displaystyle m_{2}$};
% Text Node
\draw (201,1129) node [anchor=north west][inner sep=0.75pt]    {$x_{1}$};
% Text Node
\draw (332,1129) node [anchor=north west][inner sep=0.75pt]    {$x_{2}$};
% Text Node
\draw (258,1036.2) node [anchor=north west][inner sep=0.75pt]   [align=left] {$\displaystyle k_{2}$};
% Text Node
\draw (390,1036.2) node [anchor=north west][inner sep=0.75pt]   [align=left] {$\displaystyle k_{3}$};


\end{tikzpicture}}
    \end{center}
    Tổng hợp lực tác dụng lên $m_1$ và $m_2$ là
    \begin{equation}
    \left\{
    \begin{array}{ccc}
    m_1 x_1'' &=& - k_1 x_1 + k_2 (x_2 - x_1) \\
    m_2 x_2'' &=& - k_3 x_2 + k_2 (x_1 - x_2).
    \end{array}
    \right.
    \label{3.1_1}
    \end{equation}
\end{frame}

\begin{frame}{2 khối lượng - 3 lò xo}
    Ta xét trường hợp đơn giản $k_1 = k_3$, $m_1 = m_2 = m$. 

    Cách 1: Thay trường hợp trên vào, ta thu được hệ
    \begin{equation}
    \left\{
    \begin{array}{ccc}
    m_1 x_1'' &=& - k_1 x_1 + k_2 (x_2 - x_1) \\
    m_2 x_2'' &=& - k_1 x_2 + k_2 (x_1 - x_2).
    \end{array}
    \right.
    \label{3.1_2}
    \end{equation}
    Để giải hệ phương trình trên, ta lấy lần lượt tính $(x_1'' - x_2'')$ và $(x_1''+x_2'')$.
    \begin{equation}
    \left\{
        \begin{array}{ccc}
        m(x_1''+x_2'') &=& -k_1(x_1+x_2) \\
        m(x_1''-x_2'') &=& -(k_1+k_2)(x_1 - x_2).
        \end{array}
    \right.
    \end{equation}
    
\end{frame}