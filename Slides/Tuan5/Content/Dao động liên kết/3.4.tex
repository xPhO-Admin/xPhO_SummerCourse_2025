\begin{frame}{Ý tưởng chính}
    Để nó có thể xuất hiện phương trình giống với phương trình dao động, ta mong muốn ánh xạ $\mathbf{[M]^{-1}}\mathbf{[K]}$ sẽ "bảo toàn hệ số".
    \begin{center}
        \resizebox{0.8\linewidth}{!}{


\tikzset{every picture/.style={line width=0.75pt}} %set default line width to 0.75pt        

\begin{tikzpicture}[x=0.75pt,y=0.75pt,yscale=-1,xscale=1]
%uncomment if require: \path (0,15225); %set diagram left start at 0, and has height of 15225

%Straight Lines [id:da46971419391016966] 
\draw    (155,5164) -- (155,5061) ;
\draw [shift={(155,5059)}, rotate = 90] [color={rgb, 255:red, 0; green, 0; blue, 0 }  ][line width=0.75]    (10.93,-3.29) .. controls (6.95,-1.4) and (3.31,-0.3) .. (0,0) .. controls (3.31,0.3) and (6.95,1.4) .. (10.93,3.29)   ;
%Straight Lines [id:da03406383173301375] 
\draw    (155,5164) -- (279.5,5164) ;
\draw [shift={(281.5,5164)}, rotate = 180] [color={rgb, 255:red, 0; green, 0; blue, 0 }  ][line width=0.75]    (10.93,-3.29) .. controls (6.95,-1.4) and (3.31,-0.3) .. (0,0) .. controls (3.31,0.3) and (6.95,1.4) .. (10.93,3.29)   ;
%Straight Lines [id:da495111073609535] 
\draw    (155,5164) -- (220.57,5146.52) ;
\draw [shift={(222.5,5146)}, rotate = 165.07] [color={rgb, 255:red, 0; green, 0; blue, 0 }  ][line width=0.75]    (10.93,-3.29) .. controls (6.95,-1.4) and (3.31,-0.3) .. (0,0) .. controls (3.31,0.3) and (6.95,1.4) .. (10.93,3.29)   ;
%Curve Lines [id:da4396149957962707] 
\draw    (300,5039.5) .. controls (325.71,5023.01) and (371.17,5021.57) .. (395.32,5037.93) ;
\draw [shift={(397.5,5039.5)}, rotate = 217.45] [fill={rgb, 255:red, 0; green, 0; blue, 0 }  ][line width=0.08]  [draw opacity=0] (8.93,-4.29) -- (0,0) -- (8.93,4.29) -- cycle    ;
%Straight Lines [id:da12429100250589431] 
\draw    (384,5165) -- (384,5062) ;
\draw [shift={(384,5060)}, rotate = 90] [color={rgb, 255:red, 0; green, 0; blue, 0 }  ][line width=0.75]    (10.93,-3.29) .. controls (6.95,-1.4) and (3.31,-0.3) .. (0,0) .. controls (3.31,0.3) and (6.95,1.4) .. (10.93,3.29)   ;
%Straight Lines [id:da011617708076129607] 
\draw    (384,5165) -- (508.5,5165) ;
\draw [shift={(510.5,5165)}, rotate = 180] [color={rgb, 255:red, 0; green, 0; blue, 0 }  ][line width=0.75]    (10.93,-3.29) .. controls (6.95,-1.4) and (3.31,-0.3) .. (0,0) .. controls (3.31,0.3) and (6.95,1.4) .. (10.93,3.29)   ;
%Straight Lines [id:da8611613220381794] 
\draw    (384,5165) -- (449.57,5147.52) ;
\draw [shift={(451.5,5147)}, rotate = 165.07] [color={rgb, 255:red, 0; green, 0; blue, 0 }  ][line width=0.75]    (10.93,-3.29) .. controls (6.95,-1.4) and (3.31,-0.3) .. (0,0) .. controls (3.31,0.3) and (6.95,1.4) .. (10.93,3.29)   ;

% Text Node
\draw (219,5032) node [anchor=north west][inner sep=0.75pt]  [font=\Large]  {$\mathcal{X}$};
% Text Node
\draw (212,5104) node [anchor=north west][inner sep=0.75pt]    {$[ q]_{\mathcal{X}} \ =
\left[
\begin{array}{c}
a _{1}\\
a _{2}
\end{array} 
\right]  $};
% Text Node
\draw (310,5001.5) node [anchor=north west][inner sep=0.75pt]    {$\mathbf{[M]^{-1}}\mathbf{[K]}$};
% Text Node
\draw (441,5105) node [anchor=north west][inner sep=0.75pt]    {$[ q'']_{\mathcal{X} ''} \ =\alpha \left[
\begin{array}{c}
a _{1}\\
a _{2}
\end{array} 
\right]  $};
% Text Node
\draw (459,5032) node [anchor=north west][inner sep=0.75pt]  [font=\Large]  {$\mathcal{X}^{''}$};


\end{tikzpicture}
}
    \end{center}
\end{frame}
\begin{frame}{Ý tưởng chính}
    Bây giờ, ta sẽ giả sử những điều sau
    \begin{enumerate}[\textbullet]
        \item \(x_1 = A e^{\lambda t}  ; \ x_2 = B e^{\lambda t} \Rightarrow [q]_{\mathcal{X}} = 
            \left[
            \begin{array}{c}
            A  \\
            B 
            \end{array} \right] \) \cite{morin2008introduction} \cite{taylor2005classic}
        \item Thoả điều kiện "bảo toàn hệ số".
    \end{enumerate}
    Nếu thoả điều kiện hai, thì ta có thể viết
    \begin{equation}
        [q'']_{\mathcal{X}''} = \lambda^2             
        \left[
            \begin{array}{c}
            A  \\
            B 
            \end{array} 
        \right].
        \label{eq:3.4_1}
    \end{equation}
    Đưa vào phương trình (\ref{eq:3.3_3}), ta có
    \begin{equation}
        \Big(\mathbf{[K]} - \lambda^2 \mathbf{[M]}\Big) 
        \left[
            \begin{array}{c}
            A  \\
            B 
            \end{array} 
        \right] = 0.
        \label{eq:3.4_2}
    \end{equation}
\end{frame}

\begin{frame}{Chéo hoá ma trận}
    Ta tìm hiểu về khái niệm Eigenvalue và Eigenvector. \cite{gilbert2009introduction}
    \vspace{2mm}

    Cho phép toán sau, với \(\mathbf{[A]}\) là ma trận ánh xạ tuyến tính; \(\mathbf{v}\) là vector
    \begin{equation}
        \mathbf{[A]} \mathbf{v} = \lambda \mathbf{v} \ \ \text{Hay} \ \ \Big( \mathbf{[A]} - \lambda \mathbf{[E]} \Big) \mathbf{v} = 0
        \label{eq:3.4_3}
    \end{equation}
    Eigenvector, là những vector \(\mathbf{v}\) thoả phương trình trên; Eigenvalue, là những giá trị \(\lambda\) tương ứng với vector \(\mathbf{v}\) là Eigenvector.
    \vspace{2mm}

    \textbf{Bước 1:} Tìm giá trị \(\lambda\) thoả (tính chất nghiệm không tầm thường)
    \begin{equation}
        \det \Big( \mathbf{[A]} - \lambda \mathbf{[E]} \Big) = 0.
        \label{eq:3.4_4}
    \end{equation}
\end{frame}
\begin{frame}{Chéo hoá ma trận}
    \textbf{Bước 2:} Thay các giá trị \(\lambda\) vào phương trình (???) để tìm eigenvector \(\mathbf{v}\).
    \vspace{2mm}

    \textbf{Bước 3:} Ta thu được \(\lambda_1, \lambda_2\) và các vector \(\mathbf{v_1}, \mathbf{v_2}\). Ta sẽ viết được
    \begin{equation}
        \mathbf{[A]} = \left[\mathbf{v_1} \ \mathbf{v_2}\right] 
        \left[
        \begin{array}{cc}
        \lambda_1 & 0 \\
        0 & \lambda_2 
        \end{array}
        \right] \left[\mathbf{v_1} \ \mathbf{v_2}\right]^{-1}
        \label{3.4_5}
    \end{equation}
    \begin{center}
        \begin{minipage}{0.45\linewidth}
            \textbf{Ví dụ:} Chéo hoá ma trận \(\mathbf{[B]}\).
            \begin{equation*}
                \mathbf{[B]} = 
                \left[
                \begin{array}{cc}
                    3 & -3 \\
                    2 & -4
                \end{array}
                \right]
            \end{equation*}
            \pause
            \textbf{Giải}

            B1: giải \(\det \Big(\mathbf{[B]} - \lambda \mathbf{[E]} \Big)=0\), ta có \(\lambda = 2,-3\).
        \end{minipage}
        \hspace{5mm}
        \begin{minipage}{0.45\linewidth}
            B2: Với \(\lambda = 2\), ta giải hệ \(\left[
            \begin{array}{cc}
                3 - 2 & -3 \\
                2 & -4 -2
            \end{array}\right] \left[
            \begin{array}{c}
            A \\
            B
            \end{array}\right] = 0\)
            \vspace{2mm}

            \(\rightarrow \mathbf{v_1} = \left[
            \begin{array}{c}
            A \\
            B
            \end{array}\right] = \left[
            \begin{array}{c}
            3 \\
            1
            \end{array}\right] \)
        \end{minipage}
    \end{center}
\end{frame}
\begin{frame}{Chéo hoá ma trận}
    \textbf{Bước 2:} Thay các giá trị \(\lambda\) vào phương trình (???) để tìm eigenvector \(\mathbf{v}\).
    \vspace{2mm}

    \textbf{Bước 3:} Ta thu được \(\lambda_1, \lambda_2\) và các vector \(\mathbf{v_1}, \mathbf{v_2}\). Ta sẽ viết được
    \begin{equation}
        \mathbf{[A]} = \left[\mathbf{v_1} \ \mathbf{v_2}\right] 
        \left[
        \begin{array}{cc}
        \lambda_1 & 0 \\
        0 & \lambda_2 
        \end{array}
        \right] \left[\mathbf{v_1} \ \mathbf{v_2}\right]^{-1}
        \label{eq:3.4_6}
    \end{equation}
    \begin{center}
        \begin{minipage}{0.45\linewidth}
            \textbf{Ví dụ:} Chéo hoá ma trận \(\mathbf{[B]}\).
            \begin{equation*}
                \mathbf{[B]} = 
                \left[
                \begin{array}{cc}
                    3 & -3 \\
                    2 & -4
                \end{array}
                \right]
            \end{equation*}
            \textbf{Giải}

            B1: giải \(\det \Big(\mathbf{[B]} - \lambda \mathbf{[E]} \Big)=0\), ta có \(\lambda = 2,-3\).
        \end{minipage}
        \hspace{5mm}
        \begin{minipage}{0.45\linewidth}
            B2: Tương tự, với \(\lambda = -3 \ \text{thì}\) 
            \vspace{2mm}
            
            \( \mathbf{v_2} = \left[
            \begin{array}{c}
            A \\
            B
            \end{array}\right] = \left[
            \begin{array}{c}
            1 \\
            2
            \end{array}\right]  \)

            B3: \[\mathbf{[B]} = \left[
                \begin{array}{cc}
                    3 & 1 \\
                    1 & 2
                \end{array}
                \right]                 
                \left[
                \begin{array}{cc}
                    2 & 0 \\
                    0 & -3
                \end{array}
                \right]                
                \left[
                \begin{array}{cc}
                    3 & 1 \\
                    1 & 2
                \end{array}
                \right]^{-1}
                \]
        \end{minipage}
    \end{center}
\end{frame}
\begin{frame}{Ý tưởng chính}

    Chúng ta sẽ tìm \(\lambda\) để thoả mãn phương trình (\ref{eq:3.4_2}), ta có tính chất nghiệm không tầm thường
    \begin{equation}
        \det{\Big( \mathbf{[K]} - \lambda^2 \mathbf{[M]}\Big)} = 0.
        \label{eq:3.4_7}
    \end{equation}
    Sau giải ra được \(\lambda\), ta thế chúng ngược lại để tìm vector \(q\) tương ứng. Những vector \(q\) được gọi là \textit{toạ độ trực giao}. Hoặc còn gọi là các \textbf{mode dao động}.
\end{frame}

\begin{frame}{Nghiệm tổng quát}
    Giả sử ta tìm được n vector \(q\), ứng với mỗi vector \(q_i\) có \(k_i\) trị riêng \(\lambda\) (eigenvalue). \(\lambda_{ij}\) là trị riêng thứ \(j\) ứng với vector \(q_i\).
    \begin{equation}
    q_{tq} = 
    \left[\begin{array}{c}
    x_1 \\ x_2
    \end{array}\right] \displaystyle = q_1 \sum_{i=0}^{k_1} A_{1i} e^{\lambda_{1i} t} + ... +  q_n \sum_{i=1}^{k_n} A_{ni} e^{\lambda_{ni} t}
    \label{eq:3.4_8}
    \end{equation}
\end{frame}