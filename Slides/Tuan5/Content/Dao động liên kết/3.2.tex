\begin{frame}{Ma trận của định luật II}
Xét hệ phương trình từ định luật II Newton
    \begin{equation*}
    \left\{
    \begin{array}{cccccc}
    m_1 x_1'' &=& - (k_1+k_2) x_1 &+& k_2 x_2 \\
    m_2 x_2'' &=&  k_2 x_1 &+& -(k_1+ k_3) x_2
    \end{array}
    \right.
    \end{equation*}
Ta có thể viết hệ phương trình thành
\begin{equation}
    \left[
    \begin{array}{cc}
    -(k_1+k_2) & k_2 \\
    k_2 & -(k_1+ k_3)
    \end{array}
    \right] 
    \left[
    \begin{array}{c}
    x_1 \\
    x_2
    \end{array}
    \right] =  
    \left[
    \begin{array}{cc}
    m_1 & 0 \\
    0 & m_2 
    \end{array}
    \right] 
    \left[
    \begin{array}{c}
    x_1'' \\
    x_2''
    \end{array}
    \right]
\end{equation}
Hay ta có thể viết
\begin{equation}
    [\mathbf{K}] 
    \left[
    \begin{array}{c}
    x_1 \\
    x_2
    \end{array}
    \right]= [\mathbf{M}] 
    \left[
    \begin{array}{c}
    x_1'' \\
    x_2''
    \end{array}
    \right]
\end{equation}
\end{frame}

\begin{frame}{Ma trận của định luật II}
    Vậy \([\mathbf{K}] \in L : \mathcal{X} \rightarrow \mathcal{X}''\). Còn \([\mathbf{M}]\) là ma trận đường chéo. 
    \vspace{2mm}

    Gọi vector \(q \in \mathcal{X}\) và \(q'' \in \mathcal{X}''\). Liên hệ giữa chúng là
    \begin{equation}
        [\mathbf{K}] q = [\mathbf{M}] q''.
    \end{equation}
\end{frame}