\begin{frame}{Hệ DĐ cưỡng bức}
    \begin{center}
        \resizebox{0.7\linewidth}{!}{\input{Content/Image/2.3_1}}
    \end{center}
    Lúc này hệ sẽ chịu thêm một lực cưỡng bức \(\mathbf{F_3}\). Tổng lực tác động lên vật
    \begin{equation}
        \displaystyle \mathbf{F} =  - k \mathbf{x} - b \mathbf{x'} + F_0 \cos{(\Omega t + \phi)} \mathbf{e_x}.
    \end{equation}
\end{frame}
\begin{frame}{Hệ DĐ cưỡng bức - PTVP}
    Lúc này ta sẽ đi giải phương trình vi phân
    \begin{equation}
        mx'' = -kx -bx' + F_0 \cos{(\Omega t + \phi)}
    \end{equation}
    Cụ thể, ta sẽ đi giải lần lượt nghiệm thuần nhất và nghiệm riêng.
    \begin{equation*}
        x = x_{0} + x_r.
    \end{equation*}
    Để giải nghiệm thuần nhất, ta đi giải phương trình vi phân sau
    \begin{equation*}
    x''_{tn} + {\displaystyle \frac{b}{m}} x'_{tn} + {\displaystyle \frac{k}{m}} x'_{tn} = 0.
    \end{equation*}
\end{frame}
\begin{frame}{Nghiệm riêng}
    Ta đặt
    \begin{equation*}
        x_r = A \cos{\left(\Omega t + \phi\right)} + B \sin{\left(\Omega t + \phi\right)}
    \end{equation*}
    Thế vào phương trình vi phân, ta đồng nhất \(\sin{()}\) và \(\cos{()}\) hai vế, ta có
    \begin{equation}
        \begin{array}{ccc}
        A &=& \displaystyle \frac{F_0}{m} \frac{\omega^2 - \Omega^2}{\left(\omega^2 - \Omega^2 \right)^2 + \left( c \Omega \right)^2}. 
        \\
        \\
        B &=& \displaystyle \frac{F_0}{m} \frac{c \Omega}{\left(\omega^2 - \Omega^2 \right)^2 + \left( c \Omega \right)^2}.
        \end{array}
    \end{equation}
\end{frame}