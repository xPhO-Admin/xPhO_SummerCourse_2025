\begin{frame}{PTVP bậc 2 không thuần nhất}
    Phương trình vi phân bậc 2 không thuần có dạng
    \begin{equation}
        a_0 x + a_1 x' + a_2 x'' = f(t)
        \label{eq:1.4_1}
    \end{equation}
    Nghiệm tổng quát \(x_{tq}\) có dạng
    \begin{equation*}
        x_{tq} = x_{tn} + x_{r}
    \end{equation*}
    Khi thế vào phương trình (\ref{eq:1.4_1}).
    \begin{equation*}
        (a_0 x_{tn} + a_1 x_{tn}' + a_2 x_{tn}'') + (a_0 x_r + a_1 x_r' + a_2 x_r'') = 0 + f(t).
    \end{equation*}
\end{frame}
\begin{frame}{Giải nghiệm riêng}
    \(x_r\) tuân theo đặc trưng của \(f(t)\).
    \begin{enumerate}[\textbullet]
        \item Nếu \(f(t)\) là đa thức bậc n \(\rightarrow\) x cũng có dạng đa thức bậc n.
        \item Nếu \(f(t)\) là hàm \(\cos{()}\) \(\rightarrow\) x cũng có dạng \(A_r \cos{()} + B_r \sin{()}\).
    \end{enumerate}
    \textbf{!!!} Phải tìm những hệ số.
    \vspace{2mm}
    \pause

    Ví dụ: tìm nghiệm riêng của \(a_0 x + a_1 x' + a_2 x'' = \alpha x^2 + \beta x + \gamma\)
    \pause
    \vspace{2mm}

    \begin{center}
        \begin{minipage}{0.45\linewidth}
        \underline{Giải}
        \vspace{2mm}
        
            Giả sử \(x_r = A x^2 + B x + C\), thế vô PTVP ta sẽ có dạng
            \begin{equation*}
                ()x^2 + ()x + () = \alpha x^2 + \beta x + \gamma
            \end{equation*}
        \end{minipage}
        \hspace{3mm}
        \begin{minipage}{0.45\linewidth}
            Đồng nhất hai vế
            \begin{equation*}
            \left\{
                \begin{array}{lcl}
                a_0 A &=& \alpha \\
                2a_1 A + a_0 B &=& \beta \\
                2a_2 A + a_1 B + a_0 C &=& \gamma
                \end{array}
            \right.
            \end{equation*}
        \end{minipage}
    \end{center}
    
    
\end{frame}

