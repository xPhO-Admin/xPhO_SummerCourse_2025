\begin{frame}
    \frametitle{Tiếp tuyến và tốc độ biến thiên}
  Xét đường cát tuyến của đường cong có phương trình $y=f(x)$ đi qua 2 điểm $P(a, f(a))$ và $Q(x, f(x))$ với $x\neq a$. Hệ số góc của đường cát tuyến PQ:
  \begin{equation}
      m_{PQ}=\dfrac{f(x)-f(a)}{x-a}
  \end{equation}
  \textit{(Hình vẽ)}
  \end{frame}
  \begin{frame}
  {Tiếp tuyến và tốc độ biến thiên}
  \begin{tcolorbox}[colback=blue!10, colframe=blue!50!black, title=Định nghĩa]
  \textbf{Tiếp tuyến} của đường cong $y=f(x)$ tại điểm $P(a, f(a)$ là đường thẳng đi qua $P$ với hệ số góc
  \begin{equation}
      m=\lim_{x\rightarrow a}\dfrac{f(x)-f(a)}{x-a}=f'(a)
  \end{equation}
  Nếu giới hạn này tồn tại.
  \end{tcolorbox}
  \begin{tcolorbox}[colback=blue!10, colframe=blue!50!black, title= ]
  Đạo hàm $f'(a)$ chính là hệ số góc của đường tiếp tuyến của đường cong $y=f(x)$ tại $x=a$.
  \end{tcolorbox}
    \end{frame}
  \begin{frame}{Tiếp tuyến và tốc độ biến thiên}
      Giả sử $y$ là đại lượng phụ thuộc vào một đại lượng khác $x$. Khi đó ta viết $y=f(x)$. Nếu $x$ biến thiên từ $x_1$ đến $x_2$ tương ứng với $y$ biến thiên từ $y_1$ đến $y_2$, tỷ sai phân
      \begin{equation}
          \dfrac{\Delta y}{\Delta x}=\dfrac{f(x_2)-f(x_1)}{x_2-x_1}
      \end{equation}
      được gọi là \textbf{tốc độ biến thiên trung bình} của y tương ứng với x.
  
      \begin{tcolorbox}[colback=blue!10, colframe=blue!50!black, title= ]
      Giới hạn
      \begin{equation}
          \lim_{\Delta x \rightarrow 0}\dfrac{\Delta y}{\Delta x}=\lim_{x_2 \rightarrow x_1}\dfrac{f(x_2)-f(x_1)}{x_2-x_1}=f'(x_1)
      \end{equation}
      được gọi là \textbf{tốc độ biến thiên tức thời} của $y$ tương ứng với $x$.
      \end{tcolorbox}
  \end{frame}
  \begin{frame}{Tiếp tuyến và tốc độ biến thiên}
      \begin{tcolorbox}[colback=blue!10, colframe=blue!50!black, title= ]
      Đạo hàm $f'(a)$ là tốc độ biến thiên tức thời của $y=f(x)$ tại $x=a$.
      \end{tcolorbox}
      Nếu $f(x)$ là quãng đường đi được của một vật, $x$ là thời gian đi, đạo hàm $f'(a)$ chính là \textbf{vận tốc tức thời} của vật tại thời điểm $x=a$.
  \end{frame}
  