Vật lý bắt đầu từ sự quan sát điều thay đổi và không thay đổi. Khi vị trí của hai thứ nào đó thay đổi, chúng ta, một cách tự nhiên, sẽ thấy vật nào hoàn thành sự thay đổi đó trước, hay sau. Ta cần một đại lượng đặc trưng cho sự so sánh này. Ta biết rằng có một đại lượng như thế, đó là tốc độ, được định nghĩa là tỷ số giữa khoảng cách di chuyển được và thời gian tương ứng, hay
\[v=\frac{\Delta s}{\Delta t}.\] 
Khi đó tốc độ của một vật thể đang chuyển động trong một khoảng thời gian quanh một thời điểm nào đó, quãng đường mà nó di chuyển được cũng được đo. 
\vspace{8pt}

Hãy xem xét một  vật giả tưởng chuyển động với tốc độ tuỳ ý nó, tức là có thể không đều, thay đổi theo một cách kỳ quái nào đó theo thời gian. Nghĩa là nếu trong \(1\) giây trước nó đi được \(10m\), trong \(1\) giây sau có thể chỉ là \(3m\).
