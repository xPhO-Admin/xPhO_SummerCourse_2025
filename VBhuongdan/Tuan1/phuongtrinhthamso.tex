\begin{definition}
    Giả sử hai toạ độ $x$, $y$ trên mặt phẳng toạ độ lần lượt là các hàm của một biến thứ ba, $t$ (gọi là \emph{tham số}) được biểu diễn qua các phương trình:
     \begin{equation*}   x=f(t),\ y=g(t),
\end{equation*}
    gọi là các \emph{phương trình tham số}. Mỗi một giá trị của $t$ xác định một điểm $(x;y)$. Khi tham số thay đổi, điểm $(x;y)$ thay đổi và vẽ ra một đường cong trên mặt phẳng toạ độ gọi là \emph{đường cong tham số}.
\end{definition}