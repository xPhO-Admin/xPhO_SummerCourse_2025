\documentclass[a4paper,12pt]{book}

\usepackage{lmodern}
\usepackage[utf8]{vietnam}
\usepackage{geometry}
\geometry{top=2.5cm, bottom=2.5cm, left=3cm, right=3cm}
\usepackage{titlesec}
\usepackage{setspace}

% Định dạng tiêu đề chương
\titleformat{\chapter}[display]
  {\normalfont\Large\bfseries}{\centering Tuần 1}{10pt}{\centering\Huge\bfseries}
  
\begin{document}

\chapter{Mở đầu về giải tích}


\begin{itemize}
    \item Giải tích là toán học của sự thay đổi.

    \item (lịch sử)123456

    \item (lịch sử)

    \item (lịch sử)
    \item (lịch sử)

\end{itemize}
\noindent Trong tuần 1, chúng tôi trình bày nội dung về hàm số và giới hạn của hàm số, đạo hàm cùng ứng dụng của chúng.
\newpage
\section{Tóm tắt lý thuyết}
\subsection{Hàm số và giới hạn hàm số}
úm ba la xì bùa
\section{Hướng dẫn học}
\section{Bài tập}
\end{document}
